%% Настройки документа
\documentclass [a4paper, 14pt] {article} %Параметры страницы

\usepackage{array}
\newcolumntype{M}[1]{>{\centering\arraybackslash}m{#1}}
\newcolumntype{N}{@{}m{0pt}@{}}
\usepackage{float}

%% Стандартные пакеты
\usepackage {cmap} %Поиск в пдф
\usepackage [T2A] {fontenc} %Кодировка
\usepackage [utf8] {inputenc} %Кодировка исходного текста
\usepackage [english, russian] {babel} %Локализация и переносы

%% Кириллица в формулах
\usepackage{mathtext}

%% Математические пакеты 
\usepackage{amsmath,amsfonts,amssymb,amsthm,mathtools} % AMS
\usepackage{icomma} % Умная запятая 

%% Шрифты
\usepackage{euscript} % Шрифт Евклида
\usepackage{mathrsfs} % Матшрифт

\usepackage{extsizes}
\usepackage{fancyhdr}

\usepackage{graphicx}
\usepackage{setspace}

%% Графики
\usepackage{pgfplots}
\usepgfplotslibrary{fillbetween}
\usetikzlibrary{patterns}

\usepackage{geometry}
\geometry{top=20mm}
\geometry{bottom=20mm}
\geometry{left=16mm}
\geometry{right=16mm}

\usepackage[pdf]{graphviz}
\usepackage{morewrites}

\usepackage{xpatch}
\makeatletter
\newcommand*{\addFileDependency}[1]{% argument=file name and extension
  \typeout{(#1)}
  \@addtofilelist{#1}
  \IfFileExists{#1}{}{\typeout{No file #1.}}
}
\makeatother
\xpretocmd{\digraph}{\addFileDependency{#2.dot}}{}{}

\raggedbottom

\DeclareUnicodeCharacter{2212}{-}
\begin{document}

\thispagestyle{empty}

\begin{center}
	ФЕДЕРАЛЬНОЕ ГОСУДАРСТВЕННОЕ БЮДЖЕТНОЕ ОБРАЗОВАТЕЛЬНОЕ УЧРЕЖДЕНИЕ ВЫСШЕГО ОБРАЗОВАНИЯ "НАЦИОНАЛЬНЫЙ ИССЛЕДОВАТЕЛЬСКИЙ УНИВЕРСИТЕТ "МЭИ"
\end{center}

\vspace{7cm}

\begin{center}
    \Huge Отчет

	\Large К домашней работе №1
	
	\large По теоретическим моделям вычислений
\end{center}

\vspace{2cm}

\begin{flushright}
	Выполнил
	
	Ушаков Н.А. 
	
	Студент группы А-05-19
\end{flushright}

\vfill
\begin{center}
	Москва 2022
\end{center}

\newpage

\section*{Задание 1} 
Построить детерменированный конечный автомат, распознающий язык:

\begin{enumerate}
    \item 
    $L_1 = \{\omega \in \{a,b,c\}^* \enspace | \enspace |\omega|_c = 1\}$

    Ответ:
    \begin{center}
        \digraph[scale=1] {task1ans1}{
            node [shape=point]; q0;
    	    node [shape=doublecircle]; q2;
    	    node [shape=circle];
    	    rankdir=LR; 
            q0->q1;
            q1->q2 [label="c"];
            q1->q1 [label="a,b"];
            q2->q2 [label="a,b"];
        } 
    \end{center}

    \item
    $L_2 = \{\omega \in \{a,b\}^* \enspace | \enspace |\omega|_a \le 2, |\omega|_b \ge 2\}$
    \\ \\
    Рассмотрим отдельно автоматы, распознающие языки:
    \\ \\
    $L_2 = \{\omega \in \{a,b\}^* \enspace | \enspace |\omega|_a \le 2\}$
    \begin{center}
        \digraph[scale=1] {task1ans21}{
            node [shape=point]; q0;
    	    node [shape=doublecircle]; 
    	    rankdir=LR; 
            q0->q1;
            q1->q1 [label="b"];
            q1->q2 [label="a"];
            q2->q2 [label="b"];
            q2->q3 [label="a"];
            q3->q3 [label="b"];
        } 
    \end{center}
    
    $L_2 = \{\omega \in \{a,b\}^* \enspace | \enspace |\omega|_b \ge 2\}$
    \begin{center}
        \digraph[scale=1] {task1ans22}{
            node [shape=point]; p0;
    	    node [shape=doublecircle]; p3;
    	    node [shape=circle];
    	    rankdir=LR; 
            p0->p1;
            p1->p1 [label="a"];
            p1->p2 [label="b"];
            p2->p2 [label="a"];
            p2->p3 [label="b"];
            p3->p3 [label="a,b"];
        } 
    \end{center}
    
    Прямое произведение ДКА:
    
    $\Sigma_2 = \{a,b\}, \enspace s_2 = \{q1p1\}, \enspace T_2 = \{q1p3, q2p3, q3p3\}, \enspace \delta_2 =$
    \begin{center}
        \begin{tabular}{|c|c|c|}
            \hline
            Q       & a             & b       \\ \hline
            <q1,p1> & <q2,p1>       & <q1,p2> \\ \hline
            <q1,p2> & <q2,p2>       & <q1,p3> \\ \hline
            <q1,p3> & <q2,p3>       & <q1,p3> \\ \hline
            <q2,p1> & <q3,p1>       & <q2,p2> \\ \hline
            <q2,p2> & <q3,p2>       & <q2,p3> \\ \hline
            <q2,p3> & <q3,p3>       & <q2,p3> \\ \hline
            <q3,p1> & $\varnothing$ & <q3,p2> \\ \hline
            <q3,p2> & $\varnothing$ & <q3,p3> \\ \hline
            <q3,p3> & $\varnothing$ & <q3,p3> \\ \hline
        \end{tabular}        
    \end{center}
    
    Получаем следующий ДКА.
    
    Ответ:
    \begin{center}
        \digraph[scale=0.7] {task1ans2}{
            node [shape=point]; q0p0;
    	    node [shape=doublecircle]; q1p3, q2p3, q3p3;
    	    node [shape=circle];
    	    rankdir=LR; 
            q0p0->q1p1;
            q1p1->q2p1 [label="a"];
            q1p1->q1p2 [label="b"];
            q1p2->q2p2 [label="a"];
            q1p2->q1p3 [label="b"];
            q1p3->q2p3 [label="a"];
            q1p3->q1p3 [label="b"];
            q2p1->q3p1 [label="a"];
            q2p1->q2p2 [label="b"];
            q2p2->q3p2 [label="a"];
            q2p2->q2p3 [label="b"];
            q2p3->q3p3 [label="a"];
            q2p3->q2p3 [label="b"];
            q3p1->q3p2 [label="b"];
            q3p2->q3p3 [label="b"];
            q3p3->q3p3 [label="b"];
        } 
    \end{center}
        

    \item 
    $L_3 = \{\omega \in \{a,b\}^* \enspace | \enspace |\omega|_a \ne |\omega|_b\}$
    
    В общем случае, этот язык не описывается с помощью ДКА, поскольку необходимо запоминать количество символов a и b. Однако, если a и b чередуются, то можно построить следующий автомат:
    
    Ответ:
    \begin{center}
        \digraph[scale=1] {task1ans3}{
            node [shape=point]; q0;
    	    node [shape=doublecircle]; q2 q3;
    	    node [shape=circle];
    	    rankdir=LR; 
            q0->q1;
            q1->q2 [label="a"];
            q2->q1 [label="b"];
            q1->q3 [label="b"];
            q3->q1 [label="a"];
        } 
    \end{center}
    
    \item 
    $L_4 = \{\omega \in \{a,b\}^* \enspace | \enspace \omega\omega=\omega\omega\omega\}$
    
    Данный язык описывает только пустое слово.
    
    Ответ:
    \begin{center}
        \digraph[scale=0.7] {task1ans4}{
            node [shape=point]; q0;
    	    node [shape=doublecircle];
    	    rankdir=LR; 
            q0->q1;
        } 
    \end{center}
    
\end{enumerate}

\section*{Задание 2}
Построить конечный автомат, используя прямое произведение:

\begin{enumerate}
    \item 
    $L_1 = \{\omega \in \{a,b\} \enspace | \enspace |\omega|_a \ge 2 \wedge |\omega|_b \ge 2\}$
    \\ \\
    Рассмотрим отдельно автоматы, распознающие языки:
    \\ \\
    $L_{11} = \{\omega \in \{a,b\} \enspace | \enspace |\omega|_a \ge 2\}$
    \begin{center}
        \digraph[scale=1] {task2ans11}{
            node [shape=point]; q0;
    	    node [shape=doublecircle]; q3;
    	    node [shape=circle];
    	    rankdir=LR; 
            q0->q1;
            q1->q2 [label="a"];
            q1->q1 [label="b"];
            q2->q2 [label="b"];
            q2->q3 [label="a"];
            q3->q3 [label="a,b"];
        }
    \end{center}
        
    $L_{12} = \{\omega \in \{a,b\} \enspace | \enspace |\omega|_b \ge 2\}$
    \begin{center}
        \digraph[scale=1] {task2ans12}{
            node [shape=point]; p0;
    	    node [shape=doublecircle]; p3;
    	    node [shape=circle];
    	    rankdir=LR; 
            p0->p1;
            p1->p2 [label="b"];
            p1->p1 [label="a"];
            p2->p2 [label="a"];
            p2->p3 [label="b"];
            p3->p3 [label="a,b"];
        }
    \end{center}
    
    Прямое произведение ДКА:
    
    $\Sigma_1 = \{a,b\}, \enspace s_1 = \{q1p1\}, \enspace T_1 = \{q3p3\}, \enspace \delta_1 =$
    \begin{center}
        \begin{tabular}{|c|c|c|}
            \hline
            Q       & a       & b       \\ \hline
            <q1,p1> & <q2,p1> & <q1,p2> \\ \hline
            <q1,p2> & <q2,p2> & <q1,p3> \\ \hline
            <q1,p3> & <q2,p3> & <q1,p3> \\ \hline
            <q2,p1> & <q3,p1> & <q2,p2> \\ \hline
            <q2,p2> & <q3,p2> & <q2,p3> \\ \hline
            <q2,p3> & <q3,p3> & <q2,p3>  \\ \hline
            <q3,p1> & <q3,p1> & <q3,p2> \\ \hline
            <q3,p2> & <q3,p2> & <q3,p3> \\ \hline
            <q3,p3> & <q3,p3> & <q3,p3> \\ \hline
        \end{tabular}
    \end{center}

    Получаем следующий ДКА.
    
    Ответ:
    \begin{center}
        \digraph[scale=0.7] {task2ans1}{
            node [shape=point]; q0p0;
        	node [shape=doublecircle]; q3p3;
        	node [shape=circle];
        	rankdir=LR; 
            q0p0->q1p1;
            q1p1->q2p1 [label="a"];
            q1p1->q1p2 [label="b"];
            q1p2->q2p2 [label="a"];
            q1p2->q1p3 [label="b"];
            q1p3->q2p3 [label="a"];
            q1p3->q1p3 [label="b"];
            q2p1->q3p1 [label="a"];
            q2p1->q2p2 [label="b"];
            q2p2->q3p2 [label="a"];
            q2p2->q2p3 [label="b"];
            q2p3->q3p3 [label="a"];
            q2p3->q2p3 [label="b"];
            q3p1->q3p1 [label="a"];
            q3p1->q3p2 [label="b"];
            q3p2->q3p2 [label="a"];
            q3p2->q3p3 [label="b"];
            q3p3->q3p3 [label="b,a"];
        }    
    \end{center}
    
    \item
    $L_2 = \{\omega \in \{a,b\}^* \enspace | \enspace |\omega|_a \ge 3 \wedge |\omega|_b \enspace \text{нечетное}\}$
    \\ \\
    Рассмотрим отдельно автоматы, распознающие языки:
    \\ \\
    $L_{21} = \{\omega \in \{a,b\}^* \enspace | \enspace |\omega|_a \ge 3\}$
    \begin{center}
        \digraph[scale=1] {task2ans21}{
            node [shape=point]; q0;
    	    node [shape=doublecircle]; q4;
    	    node [shape=circle];
    	    rankdir=LR; 
            q0->q1;
            q1->q1 [label="b"];
            q1->q2 [label="a"]
            q2->q2 [label="b"]
            q2->q3 [label="a"]
            q3->q3 [label="b"]
            q3->q4 [label="a"]
            q4->q4 [label="a,b"]
        }
    \end{center}
    
    $L_{22} = \{\omega \in \{a,b\}^* \enspace | \enspace |\omega|_b \enspace \text{нечетное}\}$
    \begin{center}
        \digraph[scale=1] {task2ans22}{
            node [shape=point]; p0;
    	    node [shape=doublecircle]; p2;
    	    node [shape=circle];
    	    rankdir=LR; 
            p0->p1;
            p1->p1 [label="a"];
            p1->p2 [label="b"]
            p2->p2 [label="a"]
            p2->p1 [label="b"]
        }
    \end{center}
    
    Прямое произведение ДКА:
    
    $\Sigma_2 = \{a,b\}, \enspace s_2 = \{q1p1\}, \enspace T_2 = \{q4p2\}, \enspace \delta_2 =$ 
    \begin{center}
        \begin{tabular}{|c|c|c|}
            \hline
            Q       & a       & b       \\ \hline
            <q1,p1> & <q2,p1> & <q1,p2> \\ \hline
            <q1,p2> & <q2,p2> & <q1,p1> \\ \hline
            <q2,p1> & <q3,p1> & <q2,p2> \\ \hline
            <q2,p2> & <q3,p2> & <q2,p1> \\ \hline
            <q3,p1> & <q4,p1> & <q3,p2> \\ \hline
            <q3,p2> & <q4,p2> & <q3,p1> \\ \hline
            <q4,p1> & <q4,p1> & <q4,p2> \\ \hline
            <q4,p2> & <q4,p2> & <q4,p1> \\ \hline
        \end{tabular}
    \end{center}

    Получаем следующий ДКА.
    
    Ответ:
    \begin{center}
        \digraph[scale=0.7] {task2ans2}{
            node [shape=point]; q0p0;
        	node [shape=doublecircle]; q4p2;
        	node [shape=circle];
        	rankdir=LR; 
            q0p0->q1p1;
            q1p1->q2p1 [label="a"];
            q1p1->q1p2 [label="b"];
            q1p2->q2p2 [label="a"];
            q1p2->q1p1 [label="b"];
            q2p1->q3p1 [label="a"];
            q2p1->q2p2 [label="b"];
            q2p2->q3p2 [label="a"];
            q2p2->q2p1 [label="b"];
            q3p1->q4p1 [label="a"];
            q3p1->q3p2 [label="b"];
            q3p2->q4p2 [label="a"];
            q3p2->q3p1 [label="b"];
            q4p1->q4p1 [label="a"];
            q4p1->q4p2 [label="b"];
            q4p2->q4p2 [label="a"];
            q4p2->q4p1 [label="b"];
        }    
    \end{center}
    
    \item 
    $L_3 = \{\omega \in \{a,b\}^* \enspace | \enspace |\omega|_a \enspace \text{четно} \wedge |\omega|_b \enspace \text{кратно трем}\}$
    \\ \\
    Рассмотрим отдельно автоматы, распознающие языки:
    \\ \\
    $L_{31} = \{\omega \in \{a,b\}^* \enspace | \enspace |\omega|_a \enspace \text{четно}\}$
    \begin{center}
        \digraph[scale=1] {task2ans31}{
            node [shape=point]; q0;
    	    node [shape=doublecircle]; q1;
    	    node [shape=circle];
    	    rankdir=LR; 
            q0->q1;
            q1->q1 [label="b"];
            q1->q2 [label="a"];
            q2->q1 [label="a"];
            q2->q2 [label="b"];
        }
    \end{center}    
    
    $L_{32} = \{\omega \in \{a,b\}^* \enspace | \enspace |\omega|_b \enspace \text{кратно трем}\}$
    \begin{center}
        \digraph[scale=1] {task2ans32}{
            node [shape=point]; p0;
    	    node [shape=doublecircle]; p1;
    	    node [shape=circle];
    	    rankdir=LR; 
            p0->p1;
            p1->p1 [label="a"];
            p1->p2 [label="b"];
            p2->p2 [label="a"];
            p2->p3 [label="b"];
            p3->p3 [label="a"];
            p3->p1 [label="b"];
        }
    \end{center} 
    
    Прямое произведение ДКА:
    
    $\Sigma_3 = \{a,b\}, \enspace s_3 = \{q1p1\}, \enspace T_3 = \{q1p1\}, \enspace \delta_3 =$ 
    \begin{center}
        \begin{tabular}{|c|c|c|}
            \hline
            Q       & a       & b       \\ \hline
            <q1,p1> & <q2,p1> & <q1,p2> \\ \hline
            <q1,p2> & <q2,p2> & <q1,p3> \\ \hline
            <q1,p3> & <q2,p3> & <q1,p1> \\ \hline
            <q2,p1> & <q1,p1> & <q2,p2> \\ \hline
            <q2,p2> & <q1,p2> & <q2,p3> \\ \hline
            <q2,p3> & <q1,p3> & <q2,p1> \\ \hline
        \end{tabular}
    \end{center}

    Получаем следующий ДКА.
    
    Ответ:
    \begin{center}
        \digraph[scale=0.7] {task2ans3}{
            node [shape=point]; q0p0;
        	node [shape=doublecircle]; q1p1;
        	node [shape=circle];
        	rankdir=LR; 
            q0p0->q1p1;
            q1p1->q2p1 [label="a"];
            q1p1->q1p2 [label="b"];
            q1p2->q2p2 [label="a"];
            q1p2->q1p3 [label="b"];
            q1p3->q2p3 [label="a"];
            q1p3->q1p1 [label="b"];
            q2p1->q1p1 [label="a"];
            q2p1->q2p2 [label="b"];
            q2p2->q1p2 [label="a"];
            q2p2->q2p3 [label="b"];
            q2p3->q1p3 [label="a"];
            q2p3->q2p1 [label="b"];
        } 
    \end{center}
    
    \item
    $L_4 = \overline{L_3}$
    \\ \\ 
    Необходимо заменить соответствующие терминальные состояния на нетерминальные и наоборот. Получаем следующий ДКА.
    
    Ответ:
    \begin{center}
        \digraph[scale=0.7] {task2ans4}{
            node [shape=point]; q0p0;
            node [shape=circle]; q1p1;
        	node [shape=doublecircle];
        	rankdir=LR; 
            q0p0->q1p1;
            q1p1->q2p1 [label="a"];
            q1p1->q1p2 [label="b"];
            q1p2->q2p2 [label="a"];
            q1p2->q1p3 [label="b"];
            q1p3->q2p3 [label="a"];
            q1p3->q1p1 [label="b"];
            q2p1->q1p1 [label="a"];
            q2p1->q2p2 [label="b"];
            q2p2->q1p2 [label="a"];
            q2p2->q2p3 [label="b"];
            q2p3->q1p3 [label="a"];
            q2p3->q2p1 [label="b"];
        } 
    \end{center}
    
    \item
    $L_5 = L_2 / L_3 = L_2 \cap \overline{L_3} = L_2 \cap L_4$
    \\ \\
    Для удобства поместим оба ДКА рядом и переименуем все состояния используя единичные литеры.
    \begin{center}
        \digraph[scale=0.7] {task2ans51}{
            node [shape=point]; h0;
        	node [shape=doublecircle]; h8;
        	node [shape=circle];
        	rankdir=LR; 
            h0->h1;
            h1->h2 [label="a"];
            h1->h3 [label="b"];
            h3->h4 [label="a"];
            h3->h1 [label="b"];
            h2->h5 [label="a"];
            h2->h4 [label="b"];
            h4->h6 [label="a"];
            h4->h2 [label="b"];
            h5->h7 [label="a"];
            h5->h6 [label="b"];
            h6->h8 [label="a"];
            h6->h5 [label="b"];
            h7->h7 [label="a"];
            h7->h8 [label="b"];
            h8->h8 [label="a"];
            h8->h7 [label="b"];
        }    
        \digraph[scale=0.7] {task2ans52}{
            node [shape=point]; g0;
            node [shape=circle]; g1;
        	node [shape=doublecircle];
        	rankdir=LR; 
            g0->g1;
            g1->g2 [label="a"];
            g1->g4 [label="b"];
            g4->g3 [label="a"];
            g4->g5 [label="b"];
            g5->g6 [label="a"];
            g5->g1 [label="b"];
            g2->g1 [label="a"];
            g2->g3 [label="b"];
            g3->g4 [label="a"];
            g3->g6 [label="b"];
            g6->g5 [label="a"];
            g6->g2 [label="b"];
        } 
    \end{center}
    
    Прямое произведение ДКА:
    
    $\Sigma_5 = \{a,b\}, \enspace s_5 = \{h1g1\}, \enspace T_5 = \{h8g2, h8g3, h8g4, h8g5, h8g6\}, \enspace \delta_5 =$ 
    
\end{enumerate}


\section*{Задача 3}
Построить минимальный ДКА по регулярному выражению.

\begin{enumerate}
    \item 
    $(ab + aba)^*a$
    
    
    
\end{enumerate}


\end{document}