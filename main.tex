%% Настройки документа
\documentclass[a4paper, 12pt] {article} %Параметры страницы  

\usepackage{array}
\newcolumntype{M}[1]{>{\centering\arraybackslash}m{#1}}
\newcolumntype{N}{@{}m{0pt}@{}}
\usepackage{float}

%% Стандартные пакеты
\usepackage {cmap} %Поиск в пдф
\usepackage [T2A] {fontenc} %Кодировка
\usepackage [utf8] {inputenc} %Кодировка исходного текста
\usepackage [english, russian] {babel} %Локализация и переносы

%% Кириллица в формулах
\usepackage{mathtext}

%% Математические пакеты 
\usepackage{amsmath,amsfonts,amssymb,amsthm,mathtools} % AMS
\usepackage{icomma} % Умная запятая 

%% Шрифты
\usepackage{euscript} % Шрифт Евклида
\usepackage{mathrsfs} % Матшрифт

\usepackage{extsizes}
\usepackage{fancyhdr}

\usepackage{graphicx}
\usepackage{setspace}

%% Графики
\usepackage{pgfplots}
\usepgfplotslibrary{fillbetween}
\usetikzlibrary{patterns}

\usepackage{geometry}
\geometry{top=20mm}
\geometry{bottom=24mm}
\geometry{left=16mm}
\geometry{right=18mm}

\usepackage[pdf]{graphviz}
\usepackage{morewrites}

\usepackage{psfrag}

\usepackage{xpatch}
\makeatletter
\newcommand*{\addFileDependency}[1]{% argument=file name and extension
  \typeout{(#1)}
  \@addtofilelist{#1}
  \IfFileExists{#1}{}{\typeout{No file #1.}}
}
\makeatother
\xpretocmd{\digraph}{\addFileDependency{#2.dot}}{}{}

\DeclareUnicodeCharacter{2212}{-}
\raggedbottom

\begin{document}

\thispagestyle{empty}

\begin{center}
	ФЕДЕРАЛЬНОЕ ГОСУДАРСТВЕННОЕ БЮДЖЕТНОЕ ОБРАЗОВАТЕЛЬНОЕ УЧРЕЖДЕНИЕ ВЫСШЕГО ОБРАЗОВАНИЯ НАЦИОНАЛЬНЫЙ ИССЛЕДОВАТЕЛЬСКИЙ УНИВЕРСИТЕТ "МЭИ"
\end{center}

\vspace{7cm}

\begin{center}
    \Huge Отчет

	\Large К домашней работе №1
	
	\Large По теоретическим моделям вычислений
\end{center}

\vspace{3cm}

\begin{flushright}
	Выполнил
	
	Ушаков Н.А. 
	
	Студент группы А-05-19
\end{flushright}

\vfill
\begin{center}
	Москва 2022
\end{center}

\newpage

\section*{Задание 1} 
Построить детерменированный конечный автомат, распознающий язык:

\begin{enumerate}
    \item 
    $L_1 = \{\omega \in \{a,b,c\}^* \enspace | \enspace |\omega|_c = 1\}$

    Ответ:
    \begin{center}
        \digraph[scale=0.8] {task1ans1}{
            node [shape=point]; q0;
    	    node [shape=doublecircle]; q2;
    	    node [shape=circle];
    	    rankdir=LR; 
            q0->q1;
            q1->q2 [label="c"];
            q1->q1 [label="a,b"];
            q2->q2 [label="a,b"];
        } 
    \end{center}

    \item
    $L_2 = \{\omega \in \{a,b\}^* \enspace | \enspace |\omega|_a \le 2, |\omega|_b \ge 2\}$
    \\ \\
    Рассмотрим отдельно автоматы, распознающие языки:
    \\ \\
    $L_{21} = \{\omega \in \{a,b\}^* \enspace | \enspace |\omega|_a \le 2\}$
    \begin{center}
        \digraph[scale=0.8] {task1ans21}{
            node [shape=point]; q0;
    	    node [shape=doublecircle]; 
    	    rankdir=LR; 
            q0->q1;
            q1->q1 [label="b"];
            q1->q2 [label="a"];
            q2->q2 [label="b"];
            q2->q3 [label="a"];
            q3->q3 [label="b"];
        } 
    \end{center}
    
    $L_{22} = \{\omega \in \{a,b\}^* \enspace | \enspace |\omega|_b \ge 2\}$
    \begin{center}
        \digraph[scale=0.8] {task1ans22}{
            node [shape=point]; p0;
    	    node [shape=doublecircle]; p3;
    	    node [shape=circle];
    	    rankdir=LR; 
            p0->p1;
            p1->p1 [label="a"];
            p1->p2 [label="b"];
            p2->p2 [label="a"];
            p2->p3 [label="b"];
            p3->p3 [label="a,b"];
        } 
    \end{center}
    
    Прямое произведение ДКА:
    
    $\Sigma_2 = \{a,b\}, \enspace s_2 = \{q1p1\}, \enspace T_2 = \{q1p3, q2p3, q3p3\}, \enspace \delta_2 =$
    \begin{center}
        \begin{tabular}{|c|c|c|}
            \hline
            Q       & a             & b       \\ \hline
            <q1,p1> & <q2,p1>       & <q1,p2> \\ \hline
            <q1,p2> & <q2,p2>       & <q1,p3> \\ \hline
            <q1,p3> & <q2,p3>       & <q1,p3> \\ \hline
            <q2,p1> & <q3,p1>       & <q2,p2> \\ \hline
            <q2,p2> & <q3,p2>       & <q2,p3> \\ \hline
            <q2,p3> & <q3,p3>       & <q2,p3> \\ \hline
            <q3,p1> & $\varnothing$ & <q3,p2> \\ \hline
            <q3,p2> & $\varnothing$ & <q3,p3> \\ \hline
            <q3,p3> & $\varnothing$ & <q3,p3> \\ \hline
        \end{tabular}        
    \end{center}
    
    Получаем следующий ДКА.
    
    Ответ:
    \begin{center}
        \digraph[scale=0.65] {task1ans2}{
            node [shape=point]; q0p0;
    	    node [shape=doublecircle]; q1p3, q2p3, q3p3;
    	    node [shape=circle];
    	    rankdir=LR; 
            q0p0->q1p1;
            q1p1->q2p1 [label="a"];
            q1p1->q1p2 [label="b"];
            q1p2->q2p2 [label="a"];
            q1p2->q1p3 [label="b"];
            q1p3->q2p3 [label="a"];
            q1p3->q1p3 [label="b"];
            q2p1->q3p1 [label="a"];
            q2p1->q2p2 [label="b"];
            q2p2->q3p2 [label="a"];
            q2p2->q2p3 [label="b"];
            q2p3->q3p3 [label="a"];
            q2p3->q2p3 [label="b"];
            q3p1->q3p2 [label="b"];
            q3p2->q3p3 [label="b"];
            q3p3->q3p3 [label="b"];
        } 
    \end{center}
        

    \item 
    $L_3 = \{\omega \in \{a,b\}^* \enspace | \enspace |\omega|_a \ne |\omega|_b\}$
    
    В общем случае, этот язык не описывается с помощью ДКА, поскольку необходимо запоминать количество символов a и b. Однако, если a и b чередуются, то можно построить следующий автомат:
    
    Ответ:
    \begin{center}
        \digraph[scale=0.68] {task1ans3}{
            node [shape=point]; q0;
    	    node [shape=doublecircle]; q2 q3;
    	    node [shape=circle];
    	    rankdir=LR; 
            q0->q1;
            q1->q2 [label="a"];
            q2->q1 [label="b"];
            q1->q3 [label="b"];
            q3->q1 [label="a"];
        } 
    \end{center}
    
    \item 
    $L_4 = \{\omega \in \{a,b\}^* \enspace | \enspace \omega\omega=\omega\omega\omega\}$
    
    Данный язык описывает только пустое слово.
    
    Ответ:
    \begin{center}
        \digraph[scale=0.65] {task1ans4}{
            node [shape=point]; q0;
    	    node [shape=doublecircle];
    	    rankdir=LR; 
            q0->q1;
        } 
    \end{center}
\end{enumerate}

\section*{Задание 2}
Построить конечный автомат, используя прямое произведение:

\begin{enumerate}
    \item 
    $L_1 = \{\omega \in \{a,b\} \enspace | \enspace |\omega|_a \ge 2 \wedge |\omega|_b \ge 2\}$
    \\ \\
    Рассмотрим отдельно автоматы, распознающие языки:
    \\ \\
    $L_{11} = \{\omega \in \{a,b\} \enspace | \enspace |\omega|_a \ge 2\}$
    \begin{center}
        \digraph[scale=0.75] {task2ans11}{
            node [shape=point]; q0;
    	    node [shape=doublecircle]; q3;
    	    node [shape=circle];
    	    rankdir=LR; 
            q0->q1;
            q1->q2 [label="a"];
            q1->q1 [label="b"];
            q2->q2 [label="b"];
            q2->q3 [label="a"];
            q3->q3 [label="a,b"];
        }
    \end{center}
        
    $L_{12} = \{\omega \in \{a,b\} \enspace | \enspace |\omega|_b \ge 2\}$
    \begin{center}
        \digraph[scale=0.75] {task2ans12}{
            node [shape=point]; p0;
    	    node [shape=doublecircle]; p3;
    	    node [shape=circle];
    	    rankdir=LR; 
            p0->p1;
            p1->p2 [label="b"];
            p1->p1 [label="a"];
            p2->p2 [label="a"];
            p2->p3 [label="b"];
            p3->p3 [label="a,b"];
        }
    \end{center}
    
    Прямое произведение ДКА:
    
    $\Sigma_1 = \{a,b\}, \enspace s_1 = \{q1p1\}, \enspace T_1 = \{q3p3\}, \enspace \delta_1 =$
    \begin{center}
        \begin{tabular}{|c|c|c|}
            \hline
            Q       & a       & b       \\ \hline
            <q1,p1> & <q2,p1> & <q1,p2> \\ \hline
            <q1,p2> & <q2,p2> & <q1,p3> \\ \hline
            <q1,p3> & <q2,p3> & <q1,p3> \\ \hline
            <q2,p1> & <q3,p1> & <q2,p2> \\ \hline
            <q2,p2> & <q3,p2> & <q2,p3> \\ \hline
            <q2,p3> & <q3,p3> & <q2,p3>  \\ \hline
            <q3,p1> & <q3,p1> & <q3,p2> \\ \hline
            <q3,p2> & <q3,p2> & <q3,p3> \\ \hline
            <q3,p3> & <q3,p3> & <q3,p3> \\ \hline
        \end{tabular}
    \end{center}

    Получаем следующий ДКА.
    
    Ответ:
    \begin{center}
        \digraph[scale=0.65] {task2ans1}{
            node [shape=point]; q0p0;
        	node [shape=doublecircle]; q3p3;
        	node [shape=circle];
        	rankdir=LR; 
            q0p0->q1p1;
            q1p1->q2p1 [label="a"];
            q1p1->q1p2 [label="b"];
            q1p2->q2p2 [label="a"];
            q1p2->q1p3 [label="b"];
            q1p3->q2p3 [label="a"];
            q1p3->q1p3 [label="b"];
            q2p1->q3p1 [label="a"];
            q2p1->q2p2 [label="b"];
            q2p2->q3p2 [label="a"];
            q2p2->q2p3 [label="b"];
            q2p3->q3p3 [label="a"];
            q2p3->q2p3 [label="b"];
            q3p1->q3p1 [label="a"];
            q3p1->q3p2 [label="b"];
            q3p2->q3p2 [label="a"];
            q3p2->q3p3 [label="b"];
            q3p3->q3p3 [label="b,a"];
        }    
    \end{center}
    
    \item
    $L_2 = \{\omega \in \{a,b\}^* \enspace | \enspace |\omega|_a \ge 3 \wedge |\omega|_b \enspace \text{нечетное}\}$
    \\ \\
    Рассмотрим отдельно автоматы, распознающие языки:
    \\ \\
    $L_{21} = \{\omega \in \{a,b\}^* \enspace | \enspace |\omega|_a \ge 3\}$
    \begin{center}
        \digraph[scale=0.7] {task2ans21}{
            node [shape=point]; q0;
    	    node [shape=doublecircle]; q4;
    	    node [shape=circle];
    	    rankdir=LR; 
            q0->q1;
            q1->q1 [label="b"];
            q1->q2 [label="a"]
            q2->q2 [label="b"]
            q2->q3 [label="a"]
            q3->q3 [label="b"]
            q3->q4 [label="a"]
            q4->q4 [label="a,b"]
        }
    \end{center}
    
    $L_{22} = \{\omega \in \{a,b\}^* \enspace | \enspace |\omega|_b \enspace \text{нечетное}\}$
    \begin{center}
        \digraph[scale=0.75] {task2ans22}{
            node [shape=point]; p0;
    	    node [shape=doublecircle]; p2;
    	    node [shape=circle];
    	    rankdir=LR; 
            p0->p1;
            p1->p1 [label="a"];
            p1->p2 [label="b"]
            p2->p2 [label="a"]
            p2->p1 [label="b"]
        }
    \end{center}
    
    Прямое произведение ДКА:
    
    $\Sigma_2 = \{a,b\}, \enspace s_2 = \{q1p1\}, \enspace T_2 = \{q4p2\}, \enspace \delta_2 =$ 
    \begin{center}
        \begin{tabular}{|c|c|c|}
            \hline
            Q       & a       & b       \\ \hline
            <q1,p1> & <q2,p1> & <q1,p2> \\ \hline
            <q1,p2> & <q2,p2> & <q1,p1> \\ \hline
            <q2,p1> & <q3,p1> & <q2,p2> \\ \hline
            <q2,p2> & <q3,p2> & <q2,p1> \\ \hline
            <q3,p1> & <q4,p1> & <q3,p2> \\ \hline
            <q3,p2> & <q4,p2> & <q3,p1> \\ \hline
            <q4,p1> & <q4,p1> & <q4,p2> \\ \hline
            <q4,p2> & <q4,p2> & <q4,p1> \\ \hline
        \end{tabular}
    \end{center}

    Получаем следующий ДКА.
    
    Ответ:
    \begin{center}
        \digraph[scale=0.68] {task2ans2}{
            node [shape=point]; q0p0;
        	node [shape=doublecircle]; q4p2;
        	node [shape=circle];
        	rankdir=LR; 
            q0p0->q1p1;
            q1p1->q2p1 [label="a"];
            q1p1->q1p2 [label="b"];
            q1p2->q2p2 [label="a"];
            q1p2->q1p1 [label="b"];
            q2p1->q3p1 [label="a"];
            q2p1->q2p2 [label="b"];
            q2p2->q3p2 [label="a"];
            q2p2->q2p1 [label="b"];
            q3p1->q4p1 [label="a"];
            q3p1->q3p2 [label="b"];
            q3p2->q4p2 [label="a"];
            q3p2->q3p1 [label="b"];
            q4p1->q4p1 [label="a"];
            q4p1->q4p2 [label="b"];
            q4p2->q4p2 [label="a"];
            q4p2->q4p1 [label="b"];
        }    
    \end{center}
    
    \item 
    $L_3 = \{\omega \in \{a,b\}^* \enspace | \enspace |\omega|_a \enspace \text{четно} \wedge |\omega|_b \enspace \text{кратно трем}\}$
    \\ \\
    Рассмотрим отдельно автоматы, распознающие языки:
    \\ \\
    $L_{31} = \{\omega \in \{a,b\}^* \enspace | \enspace |\omega|_a \enspace \text{четно}\}$
    \begin{center}
        \digraph[scale=0.75] {task2ans31}{
            node [shape=point]; q0;
    	    node [shape=doublecircle]; q1;
    	    node [shape=circle];
    	    rankdir=LR; 
            q0->q1;
            q1->q1 [label="b"];
            q1->q2 [label="a"];
            q2->q1 [label="a"];
            q2->q2 [label="b"];
        }
    \end{center}    
    
    $L_{32} = \{\omega \in \{a,b\}^* \enspace | \enspace |\omega|_b \enspace \text{кратно трем}\}$
    \begin{center}
        \digraph[scale=0.7] {task2ans32}{
            node [shape=point]; p0;
    	    node [shape=doublecircle]; p1;
    	    node [shape=circle];
    	    rankdir=LR; 
            p0->p1;
            p1->p1 [label="a"];
            p1->p2 [label="b"];
            p2->p2 [label="a"];
            p2->p3 [label="b"];
            p3->p3 [label="a"];
            p3->p1 [label="b"];
        }
    \end{center} 
    
    Прямое произведение ДКА:
    
    $\Sigma_3 = \{a,b\}, \enspace s_3 = \{q1p1\}, \enspace T_3 = \{q1p1\}, \enspace \delta_3 =$ 
    \begin{center}
        \begin{tabular}{|c|c|c|}
            \hline
            Q       & a       & b       \\ \hline
            <q1,p1> & <q2,p1> & <q1,p2> \\ \hline
            <q1,p2> & <q2,p2> & <q1,p3> \\ \hline
            <q1,p3> & <q2,p3> & <q1,p1> \\ \hline
            <q2,p1> & <q1,p1> & <q2,p2> \\ \hline
            <q2,p2> & <q1,p2> & <q2,p3> \\ \hline
            <q2,p3> & <q1,p3> & <q2,p1> \\ \hline
        \end{tabular}
    \end{center}

    Получаем следующий ДКА.
    
    Ответ:
    \begin{center}
        \digraph[scale=0.6] {task2ans3}{
            node [shape=point]; q0p0;
        	node [shape=doublecircle]; q1p1;
        	node [shape=circle];
        	rankdir=LR; 
            q0p0->q1p1;
            q1p1->q2p1 [label="a"];
            q1p1->q1p2 [label="b"];
            q1p2->q2p2 [label="a"];
            q1p2->q1p3 [label="b"];
            q1p3->q2p3 [label="a"];
            q1p3->q1p1 [label="b"];
            q2p1->q1p1 [label="a"];
            q2p1->q2p2 [label="b"];
            q2p2->q1p2 [label="a"];
            q2p2->q2p3 [label="b"];
            q2p3->q1p3 [label="a"];
            q2p3->q2p1 [label="b"];
        } 
    \end{center}
    
    \item
    $L_4 = \overline{L_3}$
    \\ \\ 
    Необходимо заменить соответствующие терминальные состояния на нетерминальные и наоборот. Получаем следующий ДКА.
    
    Ответ:
    \begin{center}
        \digraph[scale=0.6] {task2ans4}{
            node [shape=point]; q0p0;
            node [shape=circle]; q1p1;
        	node [shape=doublecircle];
        	rankdir=LR; 
            q0p0->q1p1;
            q1p1->q2p1 [label="a"];
            q1p1->q1p2 [label="b"];
            q1p2->q2p2 [label="a"];
            q1p2->q1p3 [label="b"];
            q1p3->q2p3 [label="a"];
            q1p3->q1p1 [label="b"];
            q2p1->q1p1 [label="a"];
            q2p1->q2p2 [label="b"];
            q2p2->q1p2 [label="a"];
            q2p2->q2p3 [label="b"];
            q2p3->q1p3 [label="a"];
            q2p3->q2p1 [label="b"];
        } 
    \end{center}
    
    \item
    $L_5 = L_2 / L_3 = L_2 \cap \overline{L_3} = L_2 \cap L_4$
    \\ \\
    Для удобства поместим оба ДКА рядом и переименуем все состояния используя единичные литеры.
    \begin{center}
        \digraph[scale=0.7] {task2ans51}{
            node [shape=point]; h0;
        	node [shape=doublecircle]; h8;
        	node [shape=circle];
        	rankdir=LR; 
            h0->h1;
            h1->h2 [label="a"];
            h1->h3 [label="b"];
            h3->h4 [label="a"];
            h3->h1 [label="b"];
            h2->h5 [label="a"];
            h2->h4 [label="b"];
            h4->h6 [label="a"];
            h4->h2 [label="b"];
            h5->h7 [label="a"];
            h5->h6 [label="b"];
            h6->h8 [label="a"];
            h6->h5 [label="b"];
            h7->h7 [label="a"];
            h7->h8 [label="b"];
            h8->h8 [label="a"];
            h8->h7 [label="b"];
        }    
        \digraph[scale=0.7] {task2ans52}{
            node [shape=point]; g0;
            node [shape=circle]; g1;
        	node [shape=doublecircle];
        	rankdir=LR; 
            g0->g1;
            g1->g2 [label="a"];
            g1->g4 [label="b"];
            g4->g3 [label="a"];
            g4->g5 [label="b"];
            g5->g6 [label="a"];
            g5->g1 [label="b"];
            g2->g1 [label="a"];
            g2->g3 [label="b"];
            g3->g4 [label="a"];
            g3->g6 [label="b"];
            g6->g5 [label="a"];
            g6->g2 [label="b"];
        } 
    \end{center}
    
    Прямое произведение ДКА:
    
    $\Sigma_5 = \{a,b\}, \enspace s_5 = \{h1g1\}, \enspace T_5 = \{h8g2, h8g3, h8g4, h8g5, h8g6\}, \enspace \delta_5$ 
\end{enumerate}


\section*{Задача 3}
Построить минимальный ДКА по регулярному выражению.

\begin{enumerate}
    \item 
    $(ab + aba)^*a$
    
    Построим НКА для заданного выражения:
    \begin{center}
        \digraph[scale=0.5] {task3ans11}{
            node [shape=point]; q0;
            node [shape=doublecircle]; q12;
            node [shape=circle];
        	rankdir=LR;
        	q0->q1;
        	q1->q2 [label="lamb"];
        	q2->q3 [label="lamb"];
        	q3->q4 [label="a"];
        	q4->q5 [label="b"];
        	q5->q6 [label="lamb"];
        	q2->q7 [label="lamb"];
        	q7->q8 [label="a"];
        	q8->q9 [label="b"];
        	q9->q10 [label="a"];
        	q10->q6 [label="lamb"];
        	q6->q11 [label="lamb"];
        	q6->q2 [label="lamb"];
        	q1->q11 [label="lamb"];
        	q11->q12 [label="a"];
        } 
    \end{center}
    
    Минимизируем $\lambda$-переходы в НКА:
    \begin{center}
        \digraph[scale=0.6] {task3ans12}{
            node [shape=point]; q0;
            node [shape=doublecircle]; q10;
            node [shape=circle];
        	rankdir=LR;
        	q0->q1;
        	q1->q2 [label="lamb"];
        	q2->q3 [label="a"];
        	q3->q4 [label="b"];
        	q4->q5 [label="lamb"];
        	q1->q5 [label="lamb"];
        	q1->q6 [label="lamb"];
        	q6->q7 [label="a"];
        	q7->q8 [label="b"];
        	q8->q9 [label="a"];
        	q9->q5 [label="lamb"];
        	q5->q1 [label="lamb"];
        	q5->q10 [label="a"];
        }    
    \end{center}
    
    Получаем ДКА по алгоритму Томпсона.
    
    Ответ:
    \begin{center}
        \digraph[scale=0.5] {task3ans1}{
            node [shape=point]; q0;
            node [shape=doublecircle]; q3q7q10 q3q7q9q10;
            node [shape=circle];
        	rankdir=LR;
        	q0->q1q2q5q6;
        	q1q2q5q6->q3q7q10 [label="a"];
        	q3q7q10->q4q8 [label="b"];
        	q4q8->q3q7q9q10 [label="a"];
        	q3q7q9q10->q4q8 [label="b"];
        	q3q7q9q10->q3q7q10 [label="a"];
        }    
    \end{center}
    Данный ДКА является минимальным.\\
    
    \item
    $a(a(ab)^*b)^*(ab)^*$
    
    НКА для регулярного выражения:
    \begin{center}
        \digraph[scale=0.6] {task3ans21}{
            node [shape=point]; q0;
            node [shape=doublecircle]; q2 q6 q8;
            node [shape=circle];
        	rankdir=LR;
        	q0->q1;
        	q1->q2 [label="a"];
        	q2->q3 [label="a"];
        	q3->q4 [label="a"];
        	q4->q5 [label="b"];
        	q5->q4 [label="a"];
        	q5->q6 [label="b"];
        	q6->q3 [label="a"];
        	q3->q6 [label="b"];
        	q6->q7 [label="a"];
        	q7->q8 [label="b"];
        	q8->q7 [label="a"];
        }    
    \end{center}
    
    Эквивалентный ДКА:
    \begin{center}
        \digraph[scale=0.6] {task3ans22}{
            node [shape=point]; q0;
            node [shape=doublecircle]; q2 q6 q6q8;
            node [shape=circle];
        	rankdir=LR;
        	q0->q1;
        	q1->q2 [label="a"];
        	q2->q3 [label="a"];
        	q3->q4 [label="a"];
        	q4->q5 [label="b"];
        	q5->q4 [label="a"];
        	q5->q6 [label="b"];
        	q3->q6 [label="b"];
        	q6->q3q7 [label="a"];
        	q3q7->q4 [label="a"];
        	q3q7->q6q8 [label="b"];
        	q6q8->q3q7 [label="a"];
        }    
    \end{center}
    
    Минимизируем ДКА.
    
    Ответ:
    \begin{center}
        \digraph[scale=0.5] {task3ans2}{
            node [shape=point]; q0;
            node [shape=doublecircle]; q2_q6_q6q8;
            node [shape=circle];
        	rankdir=LR;
        	q0->q1;
        	q1->q2_q6_q6q8 [label="a"];
        	q2_q6_q6q8->q3_q5_q3q7 [label="a"];
        	q3_q5_q3q7->q2_q6_q6q8 [label="b"];
        	q3_q5_q3q7->q4 [label="a"];
        	q4->q3_q5_q3q7 [label="b"];
        }      
    \end{center}
    Данный ДКА является минимальным.\\
    
    \item
    $(a + (a + b)(a + b)b)^*$
    
    Построим НКА для заданного выражения:
    \begin{center}
        \digraph[scale=0.75] {task3ans31}{
            node [shape=point]; q0;
            node [shape=doublecircle]; q1;
            node [shape=circle];
        	rankdir=LR;
        	q0->q1;
        	q1->q1 [label="a"];
        	q1->q2 [label="a,b"];
        	q2->q3 [label="a,b"];
        	q3->q1 [label="b"];
        }      
    \end{center}
    
    Эквивалентный ДКА.
    
    Ответ:
    \begin{center}
        \digraph[scale=0.68] {task3ans3}{
            node [shape=point]; q0;
            node [shape=doublecircle]; q1 q1q2 q1q3 q1q2q3;
            node [shape=circle];
        	rankdir=LR;
        	q0->q1; 
        	q1->q1q2 [label="a"];
        	q1q2->q1q3 [label="a"];
        	q1q2->q2q3 [label="b"];
        	q1q2->q1q2q3 [label="a"];
        	q1q2q3->q1q2q3 [label="a,b"];
        	q2q3->q1q3 [label="b"];
        	q2q3->q3 [label="a"];
        	q1->q2 [label="b"];
        	q2->q3 [label="a,b"];
        	q3->q1 [label="b"];
        }      
    \end{center}
    Данный ДКА является минимальным. \\
    
    \item
    $(b + c) ((ab)^*c + (ba)^*)^*$
    
    В данном случае можно сразу построить ДКА:
    \begin{center}
        \digraph[scale=0.68] {task3ans41}{
            node [shape=point]; q0;
            node [shape=doublecircle]; q2 q5 q7;
            node [shape=circle];
        	rankdir=LR;
        	q0->q1;
        	q1->q2 [label="b,c"];
        	q2->q3 [label="a"];
        	q3->q4 [label="b"];
        	q4->q3 [label="a"];
        	q4->q5 [label="c"];
        	q2->q5 [label="c"];
        	q2->q6 [label="b"];
        	q6->q7 [label="a"];
        	q7->q6 [label="b"];
        	q7->q2 [label="a"];
        	q5->q6 [label="b"];
        	q5->q2 [label="a"];
        }      
    \end{center}
    
    Минимизируем ДКА. \\
    
    \item 
    $(a + b)^+ (aa + bb + abab + baba)(a + b)^+$
    
    НКА для регулярного выражения:
    \begin{center}
        \digraph[scale=0.62] {task3ans51}{
            node [shape=point]; q0;
            node [shape=doublecircle]; q11;
            node [shape=circle];
        	rankdir=LR;
        	q0->q1;
            q1->q2 [label="a,b"];
            q2->q1 [label="a,b"];
            q2->q3 [label="a"];
            q2->q4 [label="b"];
            q3->q5 [label="b"];
            q3->q9 [label="a"];
            q4->q6 [label="a"];
            q4->q10 [label="b"];
            q5->q7 [label="a"];
            q6->q8 [label="b"];
            q7->q9 [label="b"];
            q8->q10 [label="a"];
            q9->q11 [label="a,b"];
            q10->q11 [label="a,b"];
            q11->q11 [label="a,b"];
        }      
    \end{center}
    
    Эквивалентный ДКА:
    \begin{center}
        \digraph[scale=0.22] {task3ans52}{
            node [shape=point]; q0;
            node [shape=doublecircle]; q1q3q11, q2q5q11, q1q3q7q11, q2q9q11, q2q5q9q11, q1q4q11, q2q6q11, q1q4q8q11, q2q6q10q11, q2q10q11;
            node [shape=circle];
        	rankdir=LR;
        	q0->q1;
            q1 -> q2 [label="a,b"];
            q2 -> q1q3 [label="a"];
            q2 -> q1q4 [label="b"];
            q1q3 -> q2q9 [label="a"];
            q1q3 -> q2q5 [label="b"];
            q2q5 -> q1q3q7 [label="a"];
            q2q5 -> q1q4 [label="b"];
            q1q4 -> q2q6 [label="a"];
            q1q4 -> q2q10 [label="b"];
            q2q6 -> q1q3 [label="a"];
            q2q6 -> q1q4q8 [label="b"];
            q1q4q8 -> q2q6q10 [label="a"];
            q1q4q8 -> q2q10 [label="b"];
            q2q6q10 -> q1q3q11 [label="a"];
            q2q6q10 -> q1q4q8q11 [label="b"];
            q2q10 -> q1q3q11 [label="a"];
            q2q10 -> q1q4q11 [label="b"];
            q1q3q7 -> q2q9 [label="a"];
            q1q3q7 -> q2q5q9 [label="b"];
            q2q9 -> q1q3q11 [label="a"];
            q2q9 -> q1q4q11 [label="b"];
            q1q3q11 -> q2q9q11 [label="a"];
            q1q3q11 -> q2q5q11 [label="b"];
            q2q5q11 -> q1q3q7q11 [label="a"];
            q2q5q11 -> q1q4q11 [label="b"];
            q2q5q9 -> q1q4q11 [label="b"];
            q2q5q9 -> q1q3q7q11 [label="a"];
            q2q5q9 -> q1q4q11 [label="b"];
            q1q3q7q11 -> q2q9q11 [label="a"];
            q1q3q7q11 -> q2q5q9q11 [label="b"];
            q2q9q11 -> q1q4q11 [label="b"];
            q2q5q9q11 -> q1q3q7q11 [label="a"];
            q2q5q9q11 -> q1q4q11 [label="b"];
            q1q4q11 -> q2q6q11 [label="a"];
            q1q4q11 -> q2q10q11 [label="b"];
            q2q6q11 -> q1q4q8q11 [label="b"];
            q1q4q8q11 -> q2q6q10q11 [label="a"];
            q1q4q8q11 -> q2q10q11 [label="b"];
            q2q6q10q11 -> q1q3q11 [label="a"];
            q2q6q10q11 -> q1q4q8q11 [label="b"];
            q2q10q11 -> q1q3q11 [label="a"];
            q2q10q11 -> q1q4q11 [label="b"];
        }       
    \end{center}

    Минимизируем ДКА и переобозначим эквивалентные состояния:
    \begin{align*}
        &q1 \longrightarrow h1 \\
        &q2 \longrightarrow h2 \\
        &q1q3 \longrightarrow h3 \\
        &q1q4 \longrightarrow h4 \\
        &q2q5 \longrightarrow h5 \\
        &q2q6 \longrightarrow h6 \\
        &q1q3q7, q1q4q8 \longrightarrow h7 \\
        &q2q9, q2q6q10, q2q5q9, q2q10 \longrightarrow h8 \\
        &q2q5q11, q1q3q11, q2q6q1011, q2q6q11, q1q4q8q11 \longrightarrow h9 \\
        &q1q4q11, q2q10q11, q1q3q7q11, q2q9q11, q2q5q9q11 \longrightarrow h9 
    \end{align*}
    
    Ответ:
    \begin{center}
        \digraph[scale=0.56] {task3ans5}{
            node [shape=point]; h0;
            node [shape=doublecircle]; h9;
            node [shape=circle];
        	rankdir=LR;
        	h0->h1;
        	h1 -> h2 [label="a, b"]
            h2 -> h3 [label="a"]
            h2 -> h4 [label="b"]
            h3 -> h8 [label="a"]
            h3 -> h5 [label="b"]
            h4 -> h6 [label="a"]
            h4 -> h8 [label="b"]
            h5 -> h7 [label="a"]
            h5 -> h4 [label="b"]
            h6 -> h3 [label="a"]
            h6 -> h7 [label="b"]
            h7 -> h8 [label="a,b"]
            h8 -> h9 [label="a,b"]
            h9 -> h9 [label="a,b"]
        }       
    \end{center}
    Данный ДКА является минимальным.
\end{enumerate}

\section*{Задание №4}
Определить является ли язык регулярным или нет.

\begin{enumerate}
    \item 
    $L_1 = \{(aab)^n b(aba)^m \enspace | \enspace n\ge 0, m\ge 0\}$
    
    КА, распознающий данный язык:
    \begin{center}
        \digraph[scale=0.6] {task4ans1}{
            node [shape=point]; q0;
            node [shape=doublecircle]; q8;
            node [shape=circle];
        	rankdir=LR;
        	q0->q1;
        	q1->q2 [label="a"];
        	q2->q3 [label="a"];
        	q3->q4 [label="b"];
        	q1->q4 [label="lamb"];
        	q4->q1 [label="lamb"];
        	q4->q5 [label="b"];
        	q5->q6 [label="a"];
        	q6->q7 [label="b"];
        	q7->q8 [label="a"];
        	q5->q8 [label="lamb"];
        	q8->q5 [label="lamb"];
        }      
    \end{center}
    Соответственно, язык $L_1$ является регулярным. \\
    
    \item 
    $L_2 = \{uaav \enspace | \enspace u \in \{a,b\}^*, v \in \{a,b\}^*, |u|_b \ge |v|_a \}$
    
    Лемма о разрастании:
    \begin{center}
        $L - \text{регулярный над } \Sigma \Rightarrow \exists n \enspace \forall \omega \in L, \enspace |\omega| \ge n$ \\
        $\exists \enspace x,y,z \enspace \omega = xyz \enspace |xy| \le n \enspace y \ne \lambda \enspace \forall \enspace k \ge 0 \enspace xy^kz \in L$
    \end{center}
    
    Фиксируем $\forall n \in \mathbb{N}$. Положим $\omega = b^n aaa^n, |\omega| = 2n+2 \ge n$.
    \begin{align*}
        &\omega = xyz, |y| \ne 0, |xy| \le n : \enspace x = b^m, y = b^l, z = b^{n-m-l}aaa^n \\
        &m+l \le n, l \ne 0 \enspace \omega = xy^kz = b^{m}b^{kl}b^{n-m-l}aaa^n \\
        &\text{Положим: } k=0, \text{тогда: } \omega = b^{n-l}aaa^n \notin L_2, l \ne 0
    \end{align*}
    Лемма не выполняется, следовательно, язык $L_2$ не является регулярным. \\

    \item 
    $L_3 = \{a^m\omega \enspace | \enspace \omega \in \{a,b\}^*, 1 \le |\omega|_b \le m\}$
    
    Лемма о разрастании. Фиксируем $\forall n \in \mathbb{N}$. Положим $\omega = a^n b^n, |\omega| = 2n \ge n$.
    \begin{align*}
        &\omega = xyz, |y| \ne 0, |xy| \le n : \enspace x = a^m, y = a^l, z = a^{n-m-l}b^n \\
        &m+l \le n, l \ne 0 \enspace \omega = xy^kz = a^{m}a^{kl}a^{n-m-l}b^n \\
        &\text{Положим: } k=0, \text{тогда: } \omega = a^{n-l}b^n \notin L_3, l \ne 0
    \end{align*}
    Лемма не выполняется, следовательно, язык $L_3$ не является регулярным. \\
    
    \item 
    $L_4 = \{a^k b^m a^n \enspace | \enspace k = n \vee m > 0\}$
    
    Лемма о разрастании. Фиксируем $\forall n \in \mathbb{N}$. Положим $\omega = a^n ba^n, |\omega| = 2n+1 \ge n$.
    \begin{align*}
        &\omega = xyz, |y| \ne 0, |xy| \le n : \enspace x = a^m, y = a^l, z = a^{n-m-l}ba^n \\
        &m+l \le n, l \ne 0 \enspace \omega = xy^kz = a^{m}a^{kl}a^{n-m-l}ba^n \\
        &\text{Положим: } k=2, \text{тогда: } \omega = a^{n+l}ba^n \notin L_4, l \ne 0
    \end{align*}
    Лемма не выполняется, следовательно, язык $L_4$ не является регулярным. \\
    
    \item 
    $L_5 = \{ucv \enspace | \enspace u \in \{a,b\}^*, v \in \{a,b\}^*, u \ne v^R\}$
    
    Лемма о разрастании. Фиксируем $\forall n \in \mathbb{N}$. 
    
    Положим $\omega = (ab)^n c(ab)^n = \alpha_1\alpha_2\dots\alpha_{4n+1}, |\omega| = 4n+1 \ge n$.
    \begin{align*}
        &\omega = xyz, |y| \ne 0, |xy| \le n : \\
        &x = \alpha_1\dots\alpha_m, y = \alpha_{m+1}\dots\alpha_{m+l}, z = \alpha_{m+l+1}\dots\alpha_{4n+1} c(ab)^n \\
        &m+l \le n, l \ne 0 \\
        &\omega = xy^kz = (\alpha_1\dots\alpha_m)(\alpha_{m+1}\dots\alpha_{m+l})^k, 
        (\alpha_{m+l+1}\dots\alpha_{4n+1} c(ab)^n) \\
        &\text{Положим: } k=2, \text{тогда: } \\
        &\omega = (\alpha_1\dots\alpha_m)(\alpha_{m+1}\dots\alpha_{m+l})^2, 
        (\alpha_{m+l+1}\dots\alpha_{4n+1} c(ab)^n) \notin L_5, l \ne 0
    \end{align*}
    Лемма не выполняется, следовательно, язык $L_5$ не является регулярным. \\
\end{enumerate}

\end{document}
